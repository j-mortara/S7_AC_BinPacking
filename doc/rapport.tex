%%%%%%%%%%%%%%%%%%%%%%%%%%%%%%%%%%%%%%%%%
% Programming/Coding Assignment
% LaTeX Template
%
% This template has been downloaded from:
% http://www.latextemplates.com
%
% Original author:
% Ted Pavlic (http://www.tedpavlic.com)
%
% Note:
% The \lipsum[#] commands throughout this template generate dummy text
% to fill the template out. These commands should all be removed when 
% writing assignment content.
%
% This template uses a Perl script as an example snippet of code, most other
% languages are also usable. Configure them in the "CODE INCLUSION 
% CONFIGURATION" section.
%
%%%%%%%%%%%%%%%%%%%%%%%%%%%%%%%%%%%%%%%%%

%----------------------------------------------------------------------------------------
%	PACKAGES AND OTHER DOCUMENT CONFIGURATIONS
%----------------------------------------------------------------------------------------

\documentclass{article}
\usepackage{fancyhdr} % Required for custom headers
\usepackage{lastpage} % Required to determine the last page for the footer
\usepackage{extramarks} % Required for headers and footers
\usepackage[usenames,dvipsnames]{color} % Required for custom colors
\usepackage{graphicx} % Required to insert images
\usepackage{listings} % Required for insertion of code
\usepackage{courier} % Required for the courier font
\usepackage{lipsum} % Used for inserting dummy 'Lorem ipsum' text into the template
\usepackage[french]{babel}
\usepackage[T1]{fontenc}

% Margins
\topmargin=-0.45in
\evensidemargin=0in
\oddsidemargin=0in
\textwidth=6.5in
\textheight=9.0in
\headsep=0.25in

\linespread{1.1} % Line spacing

% Set up the header and footer
\pagestyle{fancy}
\lhead{\hmwkAuthorName} % Top left header
%\chead{\hmwkClass\ (\hmwkClassInstructor\ \hmwkClassTime): \hmwkTitle} % Top center head
\rhead{\firstxmark} % Top right header
\lfoot{\lastxmark} % Bottom left footer
\cfoot{} % Bottom center footer
\rfoot{Page\ \thepage\ sur\ \protect\pageref{LastPage}} % Bottom right footer
\renewcommand\headrulewidth{0.4pt} % Size of the header rule
\renewcommand\footrulewidth{0.4pt} % Size of the footer rule

\setlength\parindent{0pt} % Removes all indentation from paragraphs

%----------------------------------------------------------------------------------------
%	CODE INCLUSION CONFIGURATION
%----------------------------------------------------------------------------------------

\definecolor{MyDarkGreen}{rgb}{0.0,0.4,0.0} % This is the color used for comments
\lstloadlanguages{Python} % Load Perl syntax for listings, for a list of other languages supported see: ftp://ftp.tex.ac.uk/tex-archive/macros/latex/contrib/listings/listings.pdf
\lstset{language=Python, % Use Perl in this example
        frame=single, % Single frame around code
        basicstyle=\small\ttfamily, % Use small true type font
        keywordstyle=[1]\color{Blue}\bf, % Perl functions bold and blue
        keywordstyle=[2]\color{Purple}, % Perl function arguments purple
        keywordstyle=[3]\color{Blue}\underbar, % Custom functions underlined and blue
        identifierstyle=, % Nothing special about identifiers
        commentstyle=\usefont{T1}{pcr}{m}{sl}\color{MyDarkGreen}\small, % Comments small dark green courier font
        stringstyle=\color{Purple}, % Strings are purple
        showstringspaces=false, % Don't put marks in string spaces
        tabsize=5, % 5 spaces per tab
        %
        % Put standard Perl functions not included in the default language here
        morekeywords={rand},
        %
        % Put Perl function parameters here
        morekeywords=[2]{on, off, interp},
        %
        % Put user defined functions here
        morekeywords=[3]{test},
       	%
        morecomment=[l][\color{Blue}]{...}, % Line continuation (...) like blue comment
        numbers=left, % Line numbers on left
        firstnumber=1, % Line numbers start with line 1
        numberstyle=\tiny\color{Blue}, % Line numbers are blue and small
        stepnumber=5, % Line numbers go in steps of 5
        columns=fullflexible,
        keepspaces=true,
}

%----------------------------------------------------------------------------------------
%	NAME AND CLASS SECTION
%----------------------------------------------------------------------------------------

\newcommand{\hmwkTitle}{Miniprojet 2} % Assignment title
\newcommand{\hmwkDueDate}{Lundi 8 janvier 2018} % Due date
\newcommand{\hmwkClass}{Algorithmique et Complexité} % Course/class
\newcommand{\hmwkAuthorName}{Nassim Bounouas - Thomas Canava \\ Joël Cancela Vaz - Johann Mortara} % Your name

%----------------------------------------------------------------------------------------
%	TITLE PAGE
%----------------------------------------------------------------------------------------

\title{
\vspace{2in}
\textmd{\textbf{\hmwkClass :\ \hmwkTitle}}\\
\vspace{0.2in}\Large{\textit{\ \hmwkDueDate}}
\vspace{3in}
}

\author{\textbf{Nassim Bounouas - Thomas Canava - Joël Cancela Vaz - Johann Mortara}}
\date{} % Insert date here if you want it to appear below your name

%----------------------------------------------------------------------------------------

\begin{document}

\maketitle

%----------------------------------------------------------------------------------------
%	TABLE OF CONTENTS
%----------------------------------------------------------------------------------------

%\setcounter{tocdepth}{1} % Uncomment this line if you don't want subsections listed in the ToC

\newpage
\selectlanguage{french}
\tableofcontents
\newpage

\section{Préambule}
Nous prenons pour postulat que les objets proposés à l'algorithme sont de taille inférieure ou égale à celle d'une boite car dans le contraire, le problème n'aurait pas
de solutions.

Dans les sections suivantes vous trouverez les implémentations des algorithmes en \textit{python 3} sour la forme de fonctions respectant cette disposition :

\begin{lstlisting}[language=Python, frame=single]
def nomDeLalgorithme(inputs) :
    # Initialisation des boites vides (autant de boites que d'objets)
    bins = [0] * len(inputs[1])
    # Initialisation d'un index correspondant à un itérateur sur les objets
    index = 0

    ALGORITHME

    # Filtrage sur l'ensemble des boites pour ne garder que les boites utilisées
    opened_bins = list(filter(lambda x: x > 0, bins))
    # Retour des boites utilisés
    return opened_bins
\end{lstlisting}

Le paramètre \textit{inputs} correspond à un tableau :
    \begin{itemize}
        \item inputs[0] :  Taille maximale d'une boite
        \item inputs[1] :  Tableau contenant l'ensemble des objets à ranger
    \end{itemize}

%----------------------------------------------------------------------------------------
%	First Fit
%----------------------------------------------------------------------------------------

\section{First Fit}
\subsection{Description théorique de l'algorithme}
L'algorithme First Fit a pour principe de rechercher la première boîte pouvant contenir l'item courant en commençant par la plus ancienne, si aucune boîte ne peut contenir l'item, on en ouvre une nouvelle.
Complexité en $O(n^2)$

\subsection{Implémentation de l'algorithme}
\begin{lstlisting}[language=Python, frame=single]
def first_fit(inputs):
    bins = [0] * len(inputs[1])
    index = 0
    
    for item in inputs[1]:  # for each items to pack
        j = index
        while item + bins[j] > inputs[0] :
            j += 1
        bins[j] += item
       
    opened_bins = list(filter(lambda x: x > 0, bins))
    return opened_bins
\end{lstlisting}

\subsection{Explication et remarques à propos de l'implémentation}
\begin{itemize}
  \item Cet algorithme a pour avantage d'avoir théoriquement la recherche la plus rapide étant donné qu'il recherche la première boîte susceptible de contenir l'item courant.
  \item Si on considère que notre nombre de boîtes n'est pas illimité on pourrait avoir des items non alloués dans certains exemples, par exemple:
  \begin{itemize}
  \item 5 boîtes de capacité 100, 500, 200, 300, 600
  \item 4 items de poids 212, 417, 112, 426
  \end{itemize}
  Avec l'algorithme First Fit, l'item de capacité 212 serait affecté à la boîte de capacité 500, l'item 417 à la boîte 600, l'item 112 à la boîte 500 et l'item 426 ne serait pas alloué. Alors qu'avec un autre algorithme, Best Fit notamment, l'affectation reste possible avec ces 5 boîtes.
\end{itemize}

\subsection{Optimisation de l'algorithme}
Notre implémentation du first fit reprend la description donnée précédemment cependant en ajoutant une subtilité en s'inspirant de l'algorithme Next Fit, il est possible d'en réduire la complexité.
Lors que nous plaçons un objet de taille t dans une boite d'index i, il suffit de garder en mémoire ces deux informations.
Lorsque du placement de l'objet suivant, si la taille de cet objet est supérieure ou égale à la taille de l'objet placé précédemment,
il est inutile de repartir de l'index 0, il est plus intéressant de repartir de l'index où l'objet précédent a été placé (puisque les boites les plus anciennes ne pouvaient déjà plus contenir cet objet 
de taille inférieure ou égale).
Dans cette version de l'algorithme lors de la toute première itération (premier objet à placer), nous considérons un objet de taille infinie.

\begin{lstlisting}[language=Python, frame=single]
def first_fit_enhanced(inputs):
    bins = [0] * len(inputs[1])
    index = 0
    # For the first iteration we are considering the previous item as an infinite weighted item.
    previousItem = float('inf')
    previousIndex = -1
    for item in inputs[1]:
        # If the next item is lighter than the previous one, iterate from 0
        if item < previousItem:
            iterator = 0
        # Else we are starting to iterate from the last bin used (the bins before are too full to be used).
        else:
            iterator = previousIndex
        # We are iterating to a bin used
        while item + bins[iterator] > inputs[0]:
            iterator += 1
        # Store the item, save its weight and the bin index used.
        bins[iterator] += item
        previousItem = item
        previousIndex = iterator

    opened_bins = list(filter(lambda x: x > 0, bins))
    return opened_bins
\end{lstlisting}

%----------------------------------------------------------------------------------------
%	Next Fit
%----------------------------------------------------------------------------------------

% To have just one problem per page, simply put a \clearpage after each problem

\section{Next Fit}

\subsection{Description théorique de l'algorithme}
L'algorithme Next Fit est très simple, tant que les items peuvent être alloués à la boîte courante, ils sont alloués sinon on passe à la boîte suivante.
Complexité en $O(n)$

\subsection{Implémentation de l'algorithme}
\begin{lstlisting}[language=Python, frame=single]
def next_fit(inputs):
    # For n values to store, the maximum number of bins is n,
    # so we directly create a list of n bins to avoid appending items
    bins = [0] * len(inputs[1])  # Array of maximum size: number of items
    index = 0  # iterator
    for item in inputs[1]:  # for each item
        # if the capacity for the current bin is not enough for this item
        if item + bins[index] > inputs[0]:
            index += 1  # get a new bin
        bins[index] += item  # add the new item
    # We keep the bins containing items by filtering the bins list,
    # then we return the length of this filtered list.

    # used to remove all unused (=empty) bins
    opened_bins = list(filter(lambda x: x > 0, bins))
    print(len(opened_bins))
    return opened_bins
\end{lstlisting}

%----------------------------------------------------------------------------------------
%	Worst Fit
%----------------------------------------------------------------------------------------

% To have just one problem per page, simply put a \clearpage after each problem

\section{Worst Fit}

\subsection{Description théorique de l'algorithme}
L'algorithme Worst Fit se base sur la recherche de la boîte ayant la plus grande capacité disponible.
On met les objets dans la boite moins remplie parmis les boites déjà ouvertes.
On ouvre une nouvelle boites seulement si aucune des boites ouvertes précédement ne peut contenir l'objet.
Si plusieurs boites ont une capacité égale on met l'objet dans la boite ouverte en premier.
Complexité en $O(n^2)$
\subsection{Implémentation de l'algorithme}
\begin{lstlisting}[language=Python, frame=single]
def worst_fit(inputs):
    bins = [0] * len(inputs[1])
    index = 0
    for item in inputs[1]:
        j = index
        min_index = j
        min_val = inputs[0]
        # we browse the opened bins to find the emptiest one
        while j >= 0:
            # the possible equality with min_val guarantees that the earliest
            # opened bin will be picked in case of tie, as we browse the bins
            # from the latest opened one to the earliest opened one
            if bins[j] <= min_val:
                min_val = bins[j]
                min_index = j
            j -= 1
        # if the emptiest bin cannot contain the item, we create one
        if item + bins[min_index] > inputs[0]:
            index += 1
            bins[index] += item
        # if the emptiest bin can contain the item, we add it in the bin
        else:
            bins[min_index] += item
    opened_bins = list(filter(lambda x: x > 0, bins))
    return opened_bins
\end{lstlisting}

\subsection{Explication et remarques à propos de l'implémentation}


%----------------------------------------------------------------------------------------
%	Almost Worst Fit
%----------------------------------------------------------------------------------------

% To have just one problem per page, simply put a \clearpage after each problem

\section{Almost Worst Fit}

\subsection{Description théorique de l'algorithme}
Almost Worst Fit est similiaire à Worst Fit excepté qu'il recherche la boîte avec la seconde plus grande capacité disponible.

\subsection{Implémentation de l'algorithme}
\begin{lstlisting}[language=Python, frame=single]
    bins = [0] * len(inputs[1])
    index = 0
    for item in inputs[1]:
        j = index
        min_index = j
        min_val = inputs[0]
        second_min_index = None
        first_loop = True
        # we browse the opened bins to find the emptiest one
        while j >= 0:
            # the possible equality with min_val guarantees that the earliest
            # opened bin will be picked in case of tie, as we browse the bins
            # from the latest opened one to the earliest opened one
            if bins[j] <= min_val:
                # if the current bin is the first valid one encountered,
                # we don't set the second emptiest bin index
                if first_loop:
                    first_loop = False
                else:
                    second_min_index = min_index
                min_index = j
                min_val = bins[j]
            j -= 1
        # we set the emptiest bin index to the second emptiest bin index if it exists
        # What if the second-emptiest one exists but cannot contain item ?

        # Case : create another
        # Here, the next if condition will check if the second_min_index can contain
        # the item. Supposing it cannot, another bin will be opened.
        # if second_min_index is not None:
        #     min_index = second_min_index

        # Case : check if the second emptiest one can contain the item,
        # and take this one if it is the case.
        # Otherwise, keep the emptiest one.
        if second_min_index is not None and item + bins[second_min_index] < inputs[0]:
            min_index = second_min_index

        # if the bin cannot contain the item, we create one
        if item + bins[min_index] > inputs[0]:
            index += 1
            bins[index] += item
        # if the bin can contain the item, we add it in the bin
        else:
            bins[min_index] += item
    opened_bins = list(filter(lambda x: x > 0, bins))
    return opened_bins
\end{lstlisting}

\subsection{Explication et remarques à propos de l'implémentation}


%----------------------------------------------------------------------------------------
%	Best Fit
%----------------------------------------------------------------------------------------

% To have just one problem per page, simply put a \clearpage after each problem

\section{Best Fit}
\subsection{Description théorique de l'algorithme}
L'algorithme Best Fit s'appuie sur la recherche de la boîte avec la plus petite capacité disponible suffisante pour l'item courant.
Complexité en $O(n^2)$

\subsection{Implémentation de l'algorithme}
\begin{lstlisting}[language=Python, frame=single]
def best_fit(inputs):
    bins = [0] * len(inputs[1])
    index = 0
    for item in inputs[1]:
        j = index
        max_index = j
        max_val = 0
        # we browse the opened bins to find the adequate bin
        while j >= 0:
            # if the bin is fullest than the previous valid one
            # but can still contain the item,
            # it becomes the new valid one
            if max_val < bins[j] <= inputs[0] - item:
                max_val = bins[j]
                max_index = j
            j -= 1
        # If we enter this condition, this means that the bin at index max_index cannot
        # contain the item.
        # In order to enter this condition, max_index must be equal to 0.
        # Indeed, having max_index != 0 implies that we entered the precedent if
        # condition at least once,
        # meaning that the bin at index max_index can contain the item.
        # Therefore, we know by entering this condition that no bin can contain
        # the item, so we open a new one.
        if item + bins[max_index] > inputs[0]:
            index += 1
            bins[index] += item
        else:
            # we add the item in the bin
            bins[max_index] += item
    opened_bins = list(filter(lambda x: x > 0, bins))
    return opened_bins
\end{lstlisting}

\subsection{Explication et remarques à propos de l'implémentation}

%----------------------------------------------------------------------------------------
%	Statistiques
%----------------------------------------------------------------------------------------

\section{Analyse et comparaison des algorithmes}
\subsection{Exemples proposés}
\subsection{Analyse des temps d'exécutions}
\subsection{Analyse de l'espace utilisé}
\end{document}